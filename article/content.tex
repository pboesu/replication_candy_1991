\section{Introduction}

Phenology, the timing of seasonal biological phenomena, is a key aspect of plant and animal life.
It defines the timing and duration of growth and reproduction and thereby determines the ability to capture seasonally variable resources.
The study of plant and animal phenology has allowed for a better understanding of fundamental ecosystem processes such as biogeochemical cycles,
trophic interactions, animal migrations, and the response of populations and communities to global climate change,
as well as informing applications in agriculture, forestry, and public health such as varietal selection in plant and animal breeding, or integrated pest and disease management \citep{chuine2017process}.

Phenological analyses often focus on the timing of events, such as the dates of plant flowering \citep{aono2008phenological}.%, the dates of egg laying in birds \cite{shutt2019environmental}, or the dates of first or peak appearance of butterflies \citep{roy2000phenology}.
However, for many biological phenomena exact dates of particular events are more difficult to observe than the state of the system itself.
For example, repeated but sparse survey visits may enable the recording of whether a plant is in bud, flowering, or setting fruit, but not the exact dates when each of those stages was reached.
%Similarly, surveys may record the development stage of insects or amphibians, or the breeding or moult status of birds.
Such observations can be used to categorize an organism's state into discrete classes.
Further, as the progression of the annual cycle often results in a natural ordering of these classes, e.g. from least to most developed, the resulting data can be described using ordinal regression models \cite{mccullagh1980regression,agresti2010analysis}.

I here replicate a number of ordinal regression models that were developed by \citet{dennis1986stochastic} and \citet{candy1991modeling} to describe the development of the western spruce budworm \emph{Choristoneura freemani} (Lepidoptera: Tortricidae), a defoliating moth that is widespread in western North America \citep{brookes1987western}.